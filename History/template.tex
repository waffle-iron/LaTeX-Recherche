%% Informationen zusammengetragen aus dem Online Tutorial
%% LaTeX für Studierende und Wissenschaftler
%% https://www.video2brain.com/de/videotraining/latex-fuer-studierende-und-wissenschaftler
%%



%% Optionale spezielle Informationen für einige TeX Compiler 
%!TEX TS-program = pdflatex
%!TEX encoding = UTF-8 Unicode

% Dokumentklasse Standardbuch
\documentclass{book}

%% Eingabekodierung auf Unicode UTF8 setzen
%% (Muss mit Dateikodierung übereinstimmen, tut sie bei Windows 10)
\usepackage[utf8]{inputenc}

%% Neue Deutsche Rechtschreibung als Dokumentensprache festlegen
%% Wichtig für Trennung, Ligaturen, generierte Texte etc.
\usepackage[ngerman]{babel}

%% Schriftkodierung auf 8 Bit festlegen
%% Schreibt ein Ü in das generierte PDF an Stelle eines ¨U
\usepackage[T1]{fontenc}

%% Vereinfachung von Zitaten
\usepackage[autostyle,german=quotes]{csquotes}

%% Handling von Literaturzitaten
%% Volles Zitat, verweist bei Wiederholung auf vorherige Anmerkung, nutzt 'ebendar' bei aufeinanderfolgenden Wiederholungen
\usepackage[style=verbose-inote]{biblatex}

%% Einzubindende Literaturdatei
\bibliography{literatur}

%% Zu Referenzen im Dokument kann via Klick navigiert werden
\usepackage{hyperref} 


%% Metadaten für die Titelseite
\author{Marco Feltmann}
\title{Vorlage für Philomel}
\date{29. Januar 2017}

%% Zitatblock eine Schriftgröße kleiner darstellen als Fließtext benögit zwei Pakete und einen Befehl.
%% Größen relativ anpassen. \smaller, \larger etc.pp.
%% Als Erweiterung zu den fixen \small, \large etc.pp.
\usepackage{relsize}
%% Sammlung von Werkzeugen, die an beliebigen Stellen im Satz eingreifen können.

%% Um in die Makro-Engine einzugreifen, muss das Sonderzeichen @ als normaler Buchstabe behandelt werden.
\makeatletter
%% An das Makro "quote" wird der Befehl "smaller" angehängt.
%% Dadurch wird der Text relativ zur aktuellen Schriftart kleiner.
\g@addto@macro\quote{\smaller}
%% FÜr den Rest des Dokumentes soll @ wieder als Sonderzeichen behandelt werden.
\makeatother

%% Nur für dieses Dokument. :)
% Andere Schriftart
\usepackage{droid}

%% Hier fängt dann das Dokument an.
\begin{document}

%% Titelseite erstellen
\maketitle

%% Inhaltsverzeichnis erstellen
\tableofcontents
	
\chapter{Literaturzitate}

\section{Inline Zitate}

\enquote{Ein Zitat.}\autocite{Barry2010} zu dem jemand anderes \enquote{Schlaue Sachen!}\autocite{Griffiths2009} sagte.

\section{Blockzitate}

Mittels der Umgebung \enquote{quote} lässt sich ein Blockzitat einrichten, welches vom Rest des Textes durch Einrücken abgesetzt ist.

\begin{quote}
	In diesem Zitat finden wir sehr viele Sätze, Zeilen, Wörter und Zeichen sowie sogar den einen oder anderen Absatz. Erstaunlich, nicht wahr.
	
	Ja, das ist es.\autocite{Barry2010}
\end{quote}

Die Umgebung \enquote{quotation} agiert so ähnlich, übernimmt allerdings sämtliche Einstellungen, die auch für den Fließtext im Dokument gelten. Im Gegensatz zum vorherigen Beispiel werden also hier die Absätze noch einmal zusätzlich eingerückt.

Ob das so gewollt ist liegt im Auge des Betrachters.
Wie dem auch sei wird in diesem Fall die Schriftgröße nicht reduziert, da nur die \enquote{quote} Umgebung überschrieben wird, nicht die \enquote{quotation}
\begin{quotation} Auch in diesem Zitat finden wir sehr viele Sätze, Zeilen, Wörter und Zeichen sowie sogar den einen oder anderen Absatz. 
	
	Allerdings übernimmt diese Zitatumgebung die Voreinstellungen aus dem Fließtext, was mitunter nicht so gern gesehen wird und entsprechend vermieden werden könnte.\autocite{Barry2010}
\end{quotation}

%% Falls gewünscht lassen sich auch Verzeichnisse über Graphiken und Tabellen generieren.
% \listoffigures
% \listoftables

%% Falls eine Angabe zur Bibliographie gewünsch wird	
% \printbibliography
	
\end{document}